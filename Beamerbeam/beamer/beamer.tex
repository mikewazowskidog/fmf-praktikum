\documentclass{beamer}

\usepackage[utf8]{inputenc}
\usepackage[T1]{fontenc}
\usepackage[slovene]{babel}
\usepackage{lmodern}

\usepackage{array}

\usetheme{Berlin}
\usecolortheme{default}
\beamertemplatenavigationsymbolsempty
\useinnertheme[shadows]{rounded}
\useoutertheme{infolines}

\usepackage{palatino}
\usefonttheme{serif}

\newtheorem{definicija}{Definicija}
\newtheorem{izrek}{Izrek}

\begin{document}

% ===================================================================

\title{Priprava prosojnic v LaTeX-u}
\subtitle{Uporaba paketa beamer}
\author{Mihael Jošua Vlaisavljevič}
\institute{FMF Fakulteta za matematiko in fiziko}

% -------------------------------------------------------------------

   \begin{frame}{Kratek pregled}
\tableofcontents[pausesections]
   \end{frame}
% ===================================================================

\section{Razporeditev vsebine}

% -------------------------------------------------------------------

   \begin{frame}{Naštevanje}
   Za naštevanje lahko uporabimo okolje itemize:
      Prva točka.
      Druga točka.
      Tretja točka.
   ali pa okolje enumerate:
      Prva točka.
      Druga točka.
      Tretja točka.
   \end{frame}
% -------------------------------------------------------------------

   \begin{frame}{Bloki z naslovom}
   Dele besedila lahko zapišemo v bloke.
   Uporabimo okolja \texttt{block}, \texttt{exampleblock}, \texttt{alertblock}.
   Za parameter okolja napišemo naslov bloka.
   \begin{block}{Opomba}
      Tako je videti block z naslovom.
   \end{block}
   \begin{exampleblock}{Primer}
      Tako je videti exampleblock z naslovom.
   \end{exampleblock}
   \begin{alertblock}{Opozorilo}
      Tako je videti alertblock z naslovom.
   \end{alertblock}
\end{frame}
% -------------------------------------------------------------------

   Bloki brez naslova
   Blok lahko ima tudi prazen naslov.
   V takem primeru bo brez naslovne vrstice.
      Tako je videti block s praznim naslovom.
      Tako je videti exampleblock s praznim naslovom.
      Tako je videti alertblock s praznim naslovom.

% -------------------------------------------------------------------

   \begin{frame}{Stolpci}
   \begin{columns}[t]
      \begin{column}{0.5\textwidth}
         \begin{itemize}
        \item Besedilo lahko pišemo v več stolpcih.
        \item Osnovno okolje je columns.
        \item Posamezen stolpec opišemo v okolju column.
        \item Vsebina stolpca je lahko poljubna.
        \item Za primer imamo v desnem stolpcu napis v bloku in sliko sončnice.
       \end{itemize}
      \end{column}
      \end{columns}
      
      \begin{column}{0.5\textwidth}
         \centering
      \begin{exampleblock}{}
         \centering
         Slika v stolpcu.
      \end{exampleblock}
      \includegraphics{soncnica.jpg}
      \end{column}
   \end{columns}
\end{frame}{Stolpci}
      Besedilo lahko pišemo v več stolpcih.
            Osnovno okolje je columns.
            Posamezen stolpec opišemo v okolju column.
            Vsebina stolpca je lahko poljubna.
            Za primer imamo v desnem stolpcu napis v bloku in sliko sončnice.
            Slika v stolpcu.

% ===================================================================

\section{Matematične trditve}

% -------------------------------------------------------------------

   \begin{frame}{Praštevila}
      \begin{definicija}
      Praštevilo je naravno število, ki ima natanko dva delitelja.
      \end{definicija}
      \begin{exampleblock}{Zgledi}
         1 je praštevilo (ima samo enega delitelja: 1).
         2 je praštevilo (ima dva delitelja: 1 in 2).
         3 je praštevilo (ima dva delitelja: 1 in 3).
         4 ni praštevilo (ima tri delitelje: 1, 2 in 4).
      \end{exampleblock}
   \end{frame}
% -------------------------------------------------------------------

   \begin{frame}{Praštevila}
      \begin{izrek}
         Praštevil je neskončno mnogo.
      \end{izrek}
\begin{exampleblock}
         Denimo, da je praštevil končno mnogo.
         Naj bo p največje praštevilo.
         Naj bo q produkt števil 1, 2, ??, p.
         Število q+1 ni deljivo z nobenim praštevilom, torej je q+1 praštevilo.
         To je protislovje, saj je q+1>p.
         \end{exampleblock}
      
      \end{frame}
% ===================================================================

\section{Postopno odkrivanje vsebine}

% -------------------------------------------------------------------
\begin{frame}{Konstrukcija pravokotnice na premico p skozi točko T}
   \begin{columns}[c]
      \begin{column}{0.55\textwidth}
         \begin{itemize}
            \item <1->Dani sta premica p in točka T.
            \item <2->Nariši lok k s središčem v T.
            \item <3->Premico p seče v točkah A in B.
            \item <4->Nariši lok m s središčem v A.
            \item <5->Nariši lok n s središčem v B in z enakim polmerom.
            \item <6->Loka se sečeta v točki C.
            \item <7->Premica skozi točki T in C je pravokotna na p.
       \end{itemize}
      \end{column}
      \end{columns}
      
      \begin{column}{0.45\textwidth}
         \centering
      \includegraphics<1>[width=50mm]{pic1.jpg}
      \includegraphics<2>[width=50mm]{pic2.jpg}
      \includegraphics<3>[width=50mm]{pic3.jpg}
      \includegraphics<4>[width=50mm]{pic4.jpg}
      \includegraphics<5>[width=50mm]{pic5.jpg}
      \includegraphics<6>[width=50mm]{pic6.jpg}
      \includegraphics<7>[width=50mm]{pic7.jpg}
      \end{column}
   \end{columns}
\end{frame}
  

% -------------------------------------------------------------------

   \begin{frame}{Odkrivanje tabele po vrsticah}
      \begin{tabular}{c|c c c c}
      Oznaka & A & B & C & D \\ \hline
      X & 1 & 2 & 3 & 4 \\  \pause
      Y & 3 & 4 & 5 & 6 \\  \pause
      Z & 5 & 6 & 7 & 8 
      \end{tabular}
   \end{frame}

% -------------------------------------------------------------------

   \begin{frame}{Odkrivanje tabele po stolpcih}
      \begin{tabular}{c|>{\onslide<2->}c >{\onslide<3->}c >{\onslide<4->} c >{\onslide<5->}c<{\onslide}}
       Oznaka & A & B & C & D \\ \hline
       X & 1 & 2 & 3 & 4 \\  
       Y & 3 & 4 & 5 & 6 \\  
       Z & 5 & 6 & 7 & 8  
         \end{tabular}
      \end{frame}
% ===================================================================

\section{Razno}

% -------------------------------------------------------------------
\begin{frame}
\begin{tikzpicture}
   \onslide<2->
   \begin{scope}[color=black!20]
      \fill (0, 4) circle[x radius=2, y radius=1];
      \fill (0, 4) circle[x radius=2, y radius=1];
   \end{scope}

   \draw[line width=1št, dashed] (2, 0) arc[start angle=0, delta angle=180, x radius=2, y radius= 1];
\onslide
   \begin{scope}[line width=2pt]
   \draw (0, 4) circle[x radius=2, y radius=1];
   \draw (2, 0) arc[start angle=0, delta angle=-180, x radius=2, y radius= 1];
   \draw (2 , 0)--(2 , 4);
   \draw (-2 , 0)--(-2 , 4);
   \end{scope}
\onslide<3->
   \node at (2.3,2) {$h$};
   
\end{tikzpicture}
\end{frame}
% ===================================================================

\end{document}
